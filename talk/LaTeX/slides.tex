%%%%%%%%%%%%%%%%%%%%%%%%%%%%%%%%%%%%%%%%%
% Focus Beamer Presentation
% LaTeX Template
% Version 1.0 (8/8/18)
%
% This template has been downloaded from:
% http://www.LaTeXTemplates.com
%
% Original author:
% Pasquale Africa (https://github.com/elauksap/focus-beamertheme) with modifications by 
% Vel (vel@LaTeXTemplates.com)
%
% Template license:
% GNU GPL v3.0 License
%
% Important note:
% The bibliography/references need to be compiled with bibtex.
%
%%%%%%%%%%%%%%%%%%%%%%%%%%%%%%%%%%%%%%%%%

%----------------------------------------------------------------------------------------
%	PACKAGES AND OTHER DOCUMENT CONFIGURATIONS
%----------------------------------------------------------------------------------------

\documentclass{beamer}

\usetheme{focus} % Use the Focus theme supplied with the template
% Add option [numbering=none] to disable the footer progress bar
% Add option [numbering=fullbar] to show the footer progress bar as always full with a slide count

% Uncomment to enable the ice-blue theme
%\definecolor{main}{RGB}{92, 138, 168}
%\definecolor{background}{RGB}{240, 247, 255}

%------------------------------------------------

\usepackage{booktabs} % Required for better table rules
\usepackage{animate} 
\usepackage{todonotes}
%----------------------------------------------------------------------------------------
%	 TITLE SLIDE
%----------------------------------------------------------------------------------------

\title{Competitive Online Coverage of a Mobile robot}

\subtitle{2 Online Algorithms Competitively Analysed}

\author{Karim Djemai}

% \titlegraphic{\includegraphics[scale=1.25]{Images/focuslogo.pdf}} % Optional title page image, comment this line to remove it

\institute{Universität Hamburg}

\date{30.07.2023}

%------------------------------------------------

\begin{document}

%------------------------------------------------

\begin{frame}
    \maketitle % Automatically created using the information in the commands above
\end{frame}


% Teaser
% \section{Teaser}
\begin{frame}{Teaser}
    \begin{figure}
        \todo[inline]{reactivate}
        % \animategraphics[autoplay,loop,width=0.8\linewidth]{10}{Images/cat/cat--}{0}{55}
        \caption{Not very efficient, but cute.}
    \end{figure}
\end{frame}

% Table of Contents
\begin{frame}{Outline}
    \tableofcontents[hideallsubsections]
\end{frame}

\section{Introduction}
\subsection{Definitions}
\begin{frame}{Definitions}
    \todo[inline]{add fig1}
\end{frame}

\begin{frame}{Definitions}
    \begin{block}{Definition Coverage Problem}
        \begin{itemize}
            \item let $D$ be the tool size
            \item subdivide the work-area into a grid of cells of size $D$
            \item let $n$ be the number of free cells
            \item boundary cells are cells that share at least one point with the grid boundary
            \item let $m$ be the number boundary cells
            \item let $l_\mathcal{A}$ be the length of the path, generated by Algorithm $\mathcal{A}$
            \item $l_\mathcal{A}$ defines the cost for $\mathcal{A}$
                  % grid is fixed
                  % robot has no apriori knowledge
                  % no markers are used
                  % etc
        \end{itemize}
    \end{block}
\end{frame}

\begin{frame}{Trivial Solution: DFS}
    \todo[inline]{visualize DFS}
    \begin{itemize}
        \item $l_{DFS}$ is always bounded by $2nD$
        \item STC Algorithms are bounded by $(n + m) D$
        \item Both have comp. ratio of $2 - \epsilon$
        \item in practice $m \ll n$
    \end{itemize}
\end{frame}

% \begin{frame}{Definitions}
%     \todo[inline]{add fig1}

% \end{frame}

% \subsection{Complexity}
% \begin{frame}{Complexity}
% \end{frame}
\subsection{Related Work}
\begin{frame}{Related Work}
    \todo[inline]{3-5 bullet points to contextualize this paper}
\end{frame}

\section{Spiral STC}
\subsection{Idea}
\begin{frame}{Idea}
    \begin{itemize}
        \item constructing minimal spanning trees online
        \item grid of coarser 2D cells
        \item traversal of spanning tree edges
        \item internal covering of 2D cells
        \item spiral-like patterns
    \end{itemize}
\end{frame}
\subsection{2D Spiral STC}
\begin{frame}{2D Spiral STC}
    \begin{block}{Recursive Function $STC1(w,x)$:}
        %     % Sensors: A position and orientation sensor. A 4-neighbors obstacle detection sensor.
        %     % Input: A starting cell S, but no a priori knowledge of the environment.
        %     % Recursive function: STC1(w, x), where x is the current cell and w the parent cell in the spanning tree. Initialization: Call STC1(Null, S), where S is the starting cell.
        %     % STC1(w, x):
        \begin{enumerate}
            \item Mark the current cell $x$ as an old cell.
            \item While $x$ has a new obstacle-free neighboring cell:
                  \begin{enumerate}
                      \item Scan for first new neighbor of $x$ in ccw order, start with parent cell $w$. Call this $y$.
                      \item Construct spanning-tree edge from $x$ to $y$.
                      \item Move to subcell of $y$ following right-side of spanning tree edges.
                      \item Execute STC1($x$, $y$).
                  \end{enumerate}
                  %               %   End of while loop.
            \item If $x \neq S$, move back from $x$ to a subcell of $w$ along the right-side of the spanning tree edges as described below.
        \end{enumerate}
    \end{block}
    \todo[inline]{shorten}
\end{frame}
\subsection{Visualization}
\begin{frame}{Visualization}
    \todo[inline]{visualize the spiral stc algorithm}
    \todo[inline]{2 stages? do i explain it directly?}
\end{frame}
\subsection{Full Spiral STC}
\begin{frame}{Full Spiral STC}
    \todo[inline]{grayed out version of 2D}
    \todo[inline]{fig 5 deformed path}
\end{frame}

\section{Scan STC}
\subsection{Differences to Spiral STC}
\begin{frame}{Differences to Spiral STC}
    \todo[inline]{grayed out version of 2D}
\end{frame}
\subsection{Visualization}
\begin{frame}{Visualization}
\end{frame}

\section{Analysis of the Algorithms}
\subsection{Bounds}
\begin{frame}{Bounds}
    \begin{itemize}
        \item $l_{2D-Spiral-STC} \leq nD$ but ignores partially occupied 2D cells
        \item $l_{Full-Spiral-STC} \leq (n + m)D$
        \item $l_{2D-Scan-STC} \leq nD$ but ignores partially occupied 2D cells
        \item $l_{Full-Scan-STC} \leq (n + m)D$
    \end{itemize}
    \todo[inline]{???}
\end{frame}
\subsection{Theorem 1}
\begin{frame}{Theorem 1}
    \begin{block}{Theorem 1}
        Spiral-STC and Scan-STC cover the work-area grid using a path of total length $l \leq (n + m)D$
    \end{block}
    Preliminaries:
    \begin{itemize}
        \item Entry Edges, Exit Edges
        \item repetitive coverages
              \begin{itemize}
                  \item intra, inter cell
              \end{itemize}
    \end{itemize}
    \todo[inline]{fig 4?}
    \todo[inline]{fig 9?}
\end{frame}

\begin{frame}{Theorem 1}
    \textbf{Claim 1: Total repetitive coverages is bounded by $m$}
    \begin{proof}
        \begin{itemize}
            \item Suffices to show for each 2D cell
            \item Consider 4 cases
                  \begin{itemize}
                      \item 4 free subcells
                            \begin{itemize}
                                \item double sided entry edge
                                \item single sided entry edge
                            \end{itemize}
                      \item 3 free subcells
                            \begin{itemize}
                                \item double sided entry edge
                                \item single sided entry edge
                            \end{itemize}
                            ...
                  \end{itemize}
        \end{itemize}
    \end{proof}
\end{frame}

\begin{frame}{Theorem 1}
    \todo[inline]{fig10 top}
\end{frame}
\begin{frame}{Theorem 1}
    \todo[inline]{fig10 middle}
\end{frame}
\begin{frame}{Theorem 1}
    \todo[inline]{fig10 bottom}
\end{frame}

\begin{frame}{Theorem 1}
    \begin{proof}
        \begin{itemize}
            \item Number of coverages is at least $n$
            \item (Claim 1) Number of repetitive coverages is at most $m$
            \item $\implies l_{STC} \leq (n+m) D$
        \end{itemize}
    \end{proof}
\end{frame}

\subsection{Optimality and Competitive Ratio}
\begin{frame}{Optimality and Competitive Ratio}
    \begin{itemize}
        \item Area of grid: $A := n \cdot D^2$
        \item Area of boundary cells: $\delta A := m \cdot D^2$
        \item $l_{opt} \geq \frac{A}{D} = \frac{nD^2}{D} = n \cdot D$
        \item $l_{stc} \leq (n + m) \cdot D$
    \end{itemize}


    $$ \frac{l_{stc}}{l_{opt}} \leq \frac{(n+m) \cdot D}{n \cdot D} = \frac{(n + m) D^2}{n D^2} = \frac{nD^2 + mD^2}{nD^2} = \frac{nD^2 \cdot (1 + \frac{mD^2}{nD^2})}{nD^2}$$
    $$= 1 + \frac{mD^2}{nD^2} = 1 + \frac{\delta A}{A} = 2 - \epsilon$$

\end{frame}

\section{Universal Lower Bound}
\begin{frame}{Universal Lower Bound}
    \todo[inline]{trivial corridor example pic fig 12}
    \todo[inline]{problems of that}
    \todo[inline]{fig 12 and 13}
\end{frame}

\subsection{Theorem 2}
\begin{frame}{Theorem 2}
    \todo[inline]{fig13}
    \begin{itemize}
        \item lower bound for any online algorithm
        \item w.l.o.g point robot on graph
        \item detection range
        \item ring environment
        \item local incisions
        \item w.l.o.g. ccw order
        \item incision special case
    \end{itemize}
\end{frame}

\begin{frame}{Theorem 2}
    \todo[inline]{fig13c}
    \begin{itemize}
        \item $H_i := $ length of $J_i$ that was covered before incision
        \item $H := \sum_{i = 1}^{k} H_i$
        \item Best case: $I_1, ... I_k$ covered first
        \item second loop: cover truncated $J_i$
        \item then through $x_i$ to $x_{i+1}$ via $I_i$
        \item $\implies l_{\mathcal{A}} \geq 2kL + 2H + 2(H + k) = 2kL + 4H + 2k$
        \item $l_{opt} \leq kL + 2(H + k)$
        \item $\implies l_{\mathcal{A}} \geq (2 - \epsilon) l_{opt}$ where $\epsilon = \frac{2k}{kL + 2(H + k)} \ll 1$ as $L \gg 1$
    \end{itemize}
    \todo[inline]{styling}
\end{frame}

\section{Conclusion}
\begin{frame}{Conclusion}
    \begin{itemize}
        \item Spiral STC and Scan STC are competitive online algorithms for the coverage problem
    \end{itemize}
\end{frame}

\begin{frame}[focus]
    Thanks!
\end{frame}

%----------------------------------------------------------------------------------------
%	 CLOSING/SUPPLEMENTARY SLIDES
%----------------------------------------------------------------------------------------

\appendix

\begin{frame}{References}
    \nocite{*} % Display all references regardless of if they were cited
    \bibliography{bibliography.bib}
    \bibliographystyle{plain}
\end{frame}

%------------------------------------------------

% \begin{frame}{Backup Slide}
%     This is a backup slide, useful to include additional materials to answer questions from the audience.
%     \vfill
%     The package \texttt{appendixnumberbeamer} is used to refrain from numbering appendix slides.
% \end{frame}

%----------------------------------------------------------------------------------------

\end{document}

%------------------------------------------------

% \begin{frame}{Simple Slide}
%     This is a simple slide.
% \end{frame}

% %------------------------------------------------
% \subsection{Subsection 1}

% \begin{frame}[plain]{Plain Slide}
%     This is a slide with the plain style and it is numbered.
% \end{frame}

% %------------------------------------------------

% \begin{frame}[t]
%     This slide has an empty title and is aligned to top.
% \end{frame}

% %------------------------------------------------

% \begin{frame}[noframenumbering]{No Slide Numbering}
%     This slide is not numbered and is citing reference \cite{knuth74}.
% \end{frame}

% %------------------------------------------------

% \begin{frame}{Typesetting and Math}
%     The packages \texttt{inputenc} and \texttt{FiraSans}\footnote{\url{https://fonts.google.com/specimen/Fira+Sans}}\textsuperscript{,}\footnote{\url{http://mozilla.github.io/Fira/}} are used to properly set the main fonts.
%     \vfill
%     This theme provides styling commands to typeset \emph{emphasized}, \alert{alerted}, \textbf{bold}, \textcolor{example}{example text}, \dots
%     \vfill
%     \texttt{FiraSans} also provides support for mathematical symbols:
%     \begin{equation*}
%         e^{i\pi} + 1 = 0.
%     \end{equation*}
% \end{frame}

%----------------------------------------------------------------------------------------
%	 SECTION 2
%----------------------------------------------------------------------------------------

%------------------------------------------------

% \begin{frame}{Columns}
% 	\begin{columns}
% 		\column{0.5\textwidth}
% 		This text appears in the left column and wraps neatly with a margin between columns.

% 		\column{0.5\textwidth}
% 		\includegraphics[width=\linewidth]{Images/placeholder.jpg}
% 	\end{columns}
% \end{frame}

%------------------------------------------------

% \begin{frame}{Lists}
% 	\begin{columns}[T, onlytextwidth] % T for top align, onlytextwidth to suppress the margin between columns
% 		\column{0.33\textwidth}
% 		Items:
% 		\begin{itemize}
% 			\item Item 1
% 			      \begin{itemize}
% 				      \item Subitem 1.1
% 				      \item Subitem 1.2
% 			      \end{itemize}
% 			\item Item 2
% 			\item Item 3
% 		\end{itemize}

% 		\column{0.33\textwidth}
% 		Enumerations:
% 		\begin{enumerate}
% 			\item First
% 			\item Second
% 			      \begin{enumerate}
% 				      \item Sub-first
% 				      \item Sub-second
% 			      \end{enumerate}
% 			\item Third
% 		\end{enumerate}

% 		\column{0.33\textwidth}
% 		Descriptions:
% 		\begin{description}
% 			\item[First] Yes.
% 			\item[Second] No.
% 		\end{description}
% 	\end{columns}
% \end{frame}

%------------------------------------------------

% \begin{frame}{Table}
% 	\begin{table}
% 		\centering % Centre the table on the slide
% 		\begin{tabular}{l c}
% 			\toprule
% 			Discipline                           & Avg. Salary       \\
% 			\toprule
% 			\textbf{Engineering}                 & \textbf{\$66,521} \\
% 			Computer Sciences                    & \$60,005          \\
% 			Mathematics and Sciences             & \$61,867          \\
% 			Business                             & \$56,720          \\
% 			Humanities \& Social Sciences        & \$56,669          \\
% 			Agriculture and Natural Resources    & \$53,565          \\
% 			Communications                       & \$51,448          \\
% 			\midrule
% 			\textbf{Average for All Disciplines} & \textbf{\$58,114} \\
% 			\bottomrule
% 		\end{tabular}
% 		\caption{Table caption}
% 	\end{table}
% \end{frame}

%------------------------------------------------
% \begin{frame}{Blocks}
%     These blocks are part of 1 slide, to be displayed consecutively.
%     \begin{block}{Block}
%         Text.
%     \end{block}
%     \pause % Automatically creates a new "page" split between the above and above + below
%     \begin{alertblock}{Alert block}
%         Alert \alert{text}.
%     \end{alertblock}
%     \pause % Automatically creates a new "page" split between the above and above + below
%     \begin{exampleblock}{Example block}
%         Example \textcolor{example}{text}.
%     \end{exampleblock}
% \end{frame}